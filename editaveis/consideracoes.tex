\chapter[Considerações finais]{CONSIDERAÇÕES FINAIS}

Com elaboração da proposta foi possível afirmar o que vem sendo ensinado durante todos os anos da graduação, a teoria é essencialmente importante para a construção de um projeto de qualidade. 

Desenvolver um \textit{software} é mais que codificar, há muita coisa entrelaçada, requisitos, gerência, verificação e validação, por exemplo. Sem código não há \textit{software}, mas sem uma base sólida de entendimento a qualidade do produto tende a ser menor. E isso não seria diferente se tratando de gamificação.

Estudar gamificação é perpassar por diversos aspectos que vão além da engenharia de \textit{software}, além da teoria e da prática intrinsecamente ligada ao desenvolvimento de softwar e entender o que há por trás da gamificação proporciona um melhor entendimento do como o produto será desenvolvido.

A proposta para a construção da ferramenta foi elaborada levando em consideração os aspectos estudados e o planejamento para a construção do Gamifier está sendo finalizado. A expectativa é de que a ferramenta seja útil, simples e de qualidade. Pretende-se que a ferramenta esteja alinhada ao seguinte pensamento de \cite{chou2015actionable}: A vida é muito curta para se desperdiçar o tempo jogando jogos ruins . 
