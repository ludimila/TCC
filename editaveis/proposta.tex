\chapter*[Proposta]{Proposta}
\addcontentsline{toc}{chapter}{Proposta}


A proposta deste trabalho é criar uma ferramenta capaz de produzir projetos de gamificação, a ferramenta servirá para automatizar o processo de criação dos projetos, simplificar e incentivar a inserção de gamificação no cotidiano das pessoas que tem interesse pelo tema e querem aplicá-lo as suas vidas. Este capítulo está estruturado da seguinte maneira: introdução, onde será feito uma contextualização sobre gamificação e construção de projetos de gamificação, processo, onde será descrito o processo de construção da ferramenta e por fim a descrição da ferramenta em si.


\section{Introdução}


Proposta do projeto - Porque a ferramenta tem que existir?
Porque ela é necessária? Qual a motivação? Porque é importante?

Estágio 1. Estabelece um contexto que ajuda os leitores a entenderem como a
pesquisa se situa num campo de estudo maior
Estágio 2. É feita uma revisão bibliográfica, ou seja, são apresentados aspectos do
problema que já foram estudados por outros pesquisadores.
Estágio 3. Indica a necessidade de mais investigação na área
Estágio 4. Indica os objetivos/propósitos do estudo
Estágio 7. (opcional) Dá uma justificativa para se empreender o estudo em questão,
afirmando o valor do trabalho
Estágio 8. (opcional) Define a estrutura do trabalho, isto é, seu outline


Gamificação é um tema Para motivar o ser humano, gamificação é um novo olhar sobre motivação e como motivar pessoas, vai além de inserir elementos de jogos em contextos fora de jogos, gamificar é trazer pra vida real o incentivo natural que se tem ao jogar jogar um jogo, ao tentar conquistar a vitória no jogo. Há algumas décadas têm-se estudado a motivação intrinsecamente ligada ao jogo e como aplicá-la em outras áreas com o mesmo sucesso. Mas motivar apenas não basta, motivação acaba, é preciso pensar maneiras de manter a motivação, criar um ciclo onde além de motivadas as pessoas permaneçam disciplinadas, mas não por obrigação, e sim porque sentem vontade de sempre seguir adiante.

Apesar da ascenção do tema na última década ainda não existem maneiras simples de criar uma gamificação, existem menos ainda maneiras simples de criar uma gamificação efetiva, que motive o usuário do início ao fim e o auxilie a  conquistar uma meta de maneira disciplinada e divertida e que não pareça uma obrigação.

Para que o nivel de conhecimento exigido atualmente para se criar uma gamificação seja diminuído. A pessoa não vai precisar estudar o framework do Yu-Kai Chou para conseguir montar uma simplificação.

Para aumentar a produtividade de usuários com grau de conhecimento sobre gamificação.

Para relacionar os cores do octalysis e descobrir quais cores funcionar melhor em conjunto.

Para relacionar as técnicas da mesma maneira que os cores foram relacionados, e identificar quais podem interagir com quais e quais devem ser evitadas.

Para melhorar a usabilidade da plataforma

Para dar origem a funifier store 
A atividade que proporciona prazer e felicidade não é a mesma para todas as pessoas, e ela pode acontecer de forma casual, através de uma combinação de fatores internos e externos. A execução de atividades que produzem a sensação de prazer oferece também ao indivíduo a sensação de descoberta. É neste crescimento contínuo da personalidade que está um dos segredos para o alcance do momento flow (TED, Mihaly Csikszentmihaly, 2014).

Apesar de ser uma ferramenta para construção de gamificações e não uma ferramenta gamificada a inserçação de dados e o seu comportamento deve ser leve e fluído, de fácil uso  e interação por parte do usuário, para que não tenha os mesmos problemas encontrados na inserção de dados feitos através das planilhas excel e não desmotive o usuário logo no inicio do processo de construção da gamificação. E porque seria contraditório construir uma ferramenta massante e arcaica dentro de um contexto de gamificação, seria iniciar todo o processo comentendo um erro.


Segundo Yu-Kai Chou \cite{arruda2007} gamificação é o ato de cuidadosamente aplicar ao mundo real e as atividades produtivas os elementos divertidos e envolventes dos jogos, o que torna o ato de gamificar mais complexo que apenas incorporar elementos de jogos em outros contextos, gamificar é mais sobre  motivar pessoas e menos sobre pontos e troféus. 

Para produzir motivação através de gamificação Chou criou o Octalysis framework, composto de oito unidades principais, cada unidade principal possui um número variável de técnicas, as técnicas de gamificação compõem internamente as unidades principais e servem de suporte para os aspectos que pretende-se motivar. As técnicas incorporam elementos de jogos para conduzir a motivação e para que as unidades principais sejam implementadas corretamente não é necessário fazer uso de todas as técnicas.

\section{Processo de construção da ferramenta}

Como ela foi/será construída?

Para que a ferramenta possa ser construída de maneira adequada e que proporcione ao usuário uma experiência bem sucedida na criação do seu projeto de gamificação, o framework octalysis foi escolhido como alicerce para a construção da mesma, as unidades principais do framework juntamente com as técnicas de gamificação existentes estão bem alinhadas com o principal próposito da ferramenta. Yukai-Chou utiliza as unidades principais do framework para motivar os usuários e as técnicas de gamificação para representar elementos de jogos, mas as técnicas de gamificação não servem apenas para representarem elementos de jogos, elas são utilizadas também para intensificar a motivação representada pela unidade principal da qual fazem parte e o emprego correto das técnicas na construção do projeto de gamificação produz um resultado mais satisfatório. Para que uma gamificação produza um bom resultado as técnicas de gamificação devem ser utilizadas de maneira a se complementarem, atualmente não há uma definição formal de um conjunto mínimo de atributos que devem estar presentes para que uma técnica seja implementada corretamente e não há um mapeamento que informe ao construtor do projeto de gamificação como as técnicas se relacionam, se é possível implementar todas de uma vez, se a implementação de uma afeta a implementação de outra, não há uma definição de como deve ser o relacionamento entre as técnicas de gamificação e isso acarreta na dificuldade para identificar se o projeto de gamificação está sendo construído corretamente, as unidades principais do octalysis possuem uma descrição e são compostas por técnicas de gamificação, já as técnicas de gamificação possuem uma descrição, mas não é especficado qual a sua composição, para construir a ferramenta definiu-se então um conjunto de atributos que compõem as técnicas e será realizado um mapeamento que indicará se há relacionamento entre as técnicas e como isso influencia na construção do projeto de gamificação.  

O conjunto de atributos definidos para compor a estrutura interna das técnicas são alguns dos indicadores de engajamento definidos por \cite{arruda200}, Fredericks divide os engajamentos em três tipos, engajamento emocional, que trata, por exemplo, de emoções como alegria, interesse e raiva, engajamento comportamental, que trata de indicadores de conduta, tais como, esforço, atenção e persistência e engajamento congnitivo, que trata de flexibilidade para resolver um problema, concentração e domínio. Engajamento também pode ser interpretado neste contexto como envolvimento, cada tipo de engajamento possui alguns indicadores, os indicadores caracterizam os tipos de engajamento e foram escolhidos aqui também para caracterizar as técnicas de gamificação, tal como se fossem adjetivos pertencentes as técnicas de gamificação. Os indicadores escolhidos para compor a estrutura interna das técnicas são: 

\begin{itemize}

\item Envolvimento com o trabalho: que representa a quantidade de envolvimento que a técnica de gamificação exige do usuário.
\item Participação: que representa quanta participação efetiva deve existir do usuário.
\item Atenção: representa quanta atenção é exigida do usuário.
\item Persistência: representa 
\item Domínio:
\item Social:

\end{itemize}

 Além de compor a estrutura das técnicas os indicadores servem de espelho para o envolvimento do usuário com a técnica de gamificação e como técnicas e unidades principais estão ligadas, os indicadores acabam por exercer influência sobre elas também e serão ainda utilizados como critério de avaliação de gamificação. Cada um dos indicadores receberá um valor servirá de medida de aderência do indicador à técnica de gamificação, tal medida será feita de acordo com a escala de Likert (1932), a escala utilizada será composta de cinco valores que variam de um a cinco, sendo que a nota um significa que o indicador tem uma influência muito aquém na implementação da técnica de gamificação, a nota dois uma influência aquém do normal, a nota três uma influência suficiente, a nota quatro além do suficiente e uma nota cinco significa que o indicador tem uma influência muito além na implementação da técnica.  Como as técnicas representam elementos de jogos, elas compõe a parte do framework responsável por colocar a gamificação em prática e os indicadores que são melhor interpretados aqui como atributos mensuram o envolvimento do usuário com a técnica e por fim com a gamificação em si. Como cada um dos indicadores que compõem as técnicas de gamificação recebe uma nota, é possível mensurar o quanto aquele indicador está presente em uma técnica e quanta influência pode vir a  exercer sobre a mesma.


\section{Processo de interação}

Como irá funciona?

