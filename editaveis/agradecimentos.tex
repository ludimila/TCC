\begin{agradecimentos}


A Deus, por me dar capacidade de aproveitar as oportunidades que Ele me oferece e por estar comigo em todo momento. 

A minha mãe, Adontina de Jesus da Bela Cruz, que é a melhor pessoa que eu conheço, por ter me deixado chegar até aqui, por ter sido pai e mãe durante a minha vida toda, por todo o esforço, madrugadas trabalhadas e por toda a resiliência que ela possuí e tentou exaustivamente me ensinar a ter. Não enxergo um mundo em que a senhora não faça parte. Te amo, obrigado ter me ligado todos os dias durante esses cinco anos longe, sua voz salva meus dias ruins e alegra ainda mais os dias bons, eu escreveria mil linhas e não conseguiria expressar minha gratidão. 

Ao meu irmão Lucas, que me ensina todo dia que amor é amor, mesmo quando a gente briga.

A toda a minha família, que assim como minha mãe ficou na Bahia, sempre torcendo a favor e me mandando fotos das crianças, para que eu não perca a maravilha que é vê-las crescendo. A tia Ivanete Pereira, que me acolheu por um ano sem me conhecer quando me mudei pra Brasília. Sem vocês eu não teria caminhando nem um terço do que eu caminhei e ao meu pai que me ensinou que ser adulto é admitir que falhamos.

Aos meus amigos de longa, Carlos Mota, Kênia Arruda, Juliana Araújo, Mateus Gonçalves, Pedro Felipe e Sâmea Martins, que souberam conservar a amizade mesmo estando a mais de 500 km de distância ou em outro país e sabem que eu sou uma amiga relapsa, mas amo todos. 

Aos amigos que fiz aqui, Allan Pereira e Willian Gulgielmin, por toda a solidariedade, ao Pedro Alcântara, pela perspectiva de vida peculiar, a Vanusa Oliveira e Igor Ramos, por todo o suporte, a qualquer hora do dia ou da noite e ao Marcelo Araújo, todo o carinho e dedicação. Vocês são minha família e meu porto seguro nessa capital.

A todos os professores da FGA, em especial ao meu orientador, exemplo de conduta e respeito, Sérgio Antônio Andrade de Freitas, pela gentileza e a incrível habilidade de acreditar em um potencial que nem os alunos sabem que possuem. Aos professores Maurício e Milene Serrano, por me darem a oportunidade de descobrir o quão divertido é ensinar. Ao professor George Marsicano, por acreditar que seus alunos podem sempre ser melhores.


\end{agradecimentos}
