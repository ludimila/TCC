\chapter[Introdução]{Introdução}


Segundo \cite{chou2015actionable} gamificação é o ato de cuidadosamente aplicar ao mundo real e as atividades produtivas os elementos divertidos e envolventes dos jogos. O que torna o ato de gamificar mais complexo que apenas incorporar elementos de jogos em outros contextos. Gamificar é mais sobre  motivar pessoas e menos sobre pontos e troféus \cite{chou2015actionable}, \cite{zichermann2011gamification}. 

Com a intenção de construir uma ferramenta que aproxime as pessoas da gamificação surgiu a proposta deste trabalho. Constuir a partir do zero de uma ferramenta de apoio a construção de projetos de gamificação que ampliará o alcance do tema a pessoas. Aumentando a possibilidade de inserir a gamificação na vida dessas pessoas, em seu ciclo social e no ambiente em que trabalham e vivem.

A ferramenta recebeu o nome de Gamifier e não se destinará apenas à estudiosos da área de gamificação, também atenderá ao usuário não especialista. Como a intenção é aproximar as pessoas do assunto, a ferramenta Funifier foi escolhida como modelo de inspiração, por já estar consolidada no mercado realizando há algum tempo essa aproximação entre pessoas e gamificação. 


\section{Objetivos}

Os objetivos do trabalho serão detalhados em seguida, primeiro o objetivo geral em seguida os objetivos específicos.


\subsection{Objetivo Geral}

Criar uma ferramenta de apoio a criação de projetos de gamificação.

\subsection{Objetivos Específicos}

\begin{itemize}

\item  Selecionar um modelo de gamificação como base para ferramenta;
\item  Criar uma ferramenta de uso simples;
\item  Exportar os projetos de maneira que o Funifier entenda;


\end{itemize}

\section{Questão de Pesquisa}

É possível desenvolver uma ferramenta de apoio a criação de projetos de gamificação capaz de produzir projetos de qualidade e acessível a todos?


\section{Motivação}

 Autores como \cite{chou2015actionable} e \cite{mcgonigal2011reality} acreditam que o uso da gamificação pode mudar vidas e até a sociedade em que estamos inseridos. Ela tem a capacidade de transformar empresas, escolas, hospitais e umas vasta gama de ambientes, basta ser aplicada corretamente em um contexto adequado. 

 A motivação maior em desenvolver esse trabalho é poder oferecer as pessoas a oportunidade de criarem o seu próprio projeto e a partir desta iniciativa iniciarem uma mudança em suas algum aspecto de suas vidas. Segundo McGonigal, os jogos são capazes de concentrar nossa energia em algo que somos bons e apreciamos fazer, com otimismo incansável.

\section{Metodologia}

\subsection{Classificação da Pesquisa}

A realização deste trabalho é classificado como uma pesquisa exploratória, que tem como caracteríticas buscar uma abordagem do fenômeno pelo levantamento de informações que poderão levar o pesquisador a conhecer mais a seu respeito \cite{gil2010metodos}. 

\subsection{Referêncial Teórico}

O referencial teórico foi levantado utilizando livros físicos e digitais, artigos e alguns sites. O resultado das pesquisas efetuadas e que geraram o referencial teórico deste trabalho está disponível no capítulo 2.

\section{Estrutura da Monografia}

Este trabalho está fracionado em cinco capítulos, o capítulo 1 compreende a introdução, onde é feita uma contextualização sobre o tema e o trabalho. O capítulo 2, corresponde ao referencial teórico e reflete as pesquisas realizadas para possibilitar a construção do trabalho.  O capítulo 3 contém a proposta do trabalho e como ela foi elaborada. O capítulo 4, relata como ocorrerá a execução da proposta. Por fim o capítulo 5 onde estão dispostas as lições aprendidas e as considerações finais sobre o trabalho
