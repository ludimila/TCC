\chapter[Introdução]{Introdução}


Segundo \cite{mcgonigal2011reality} algo épico é aquilo que supera em muito o mediano, algo épico tem proporções heroicas. Dar as pessoas a oportunidade de criar algo épico utilizado gamificação é inspirador, tentar trazer para a vida real e para o cotidiano das pessoas a mesma motivação existente ao jogar é um dos objetivos da gamificação. Existem diversos modelos de gamificação, muitas definições, algumas ferramentas e muita disposição em colocar toda a teoria em prática. 

A criação a partir do zero de uma ferramenta de apoio a construção de projetos de gamificação amplia o alcance do tema a pessoas comuns e aumenta a possibilidade de inserir a gamificação na vida dessas pessoas, em seu ciclo social e no ambiente em que trabalham e vivem.


\section{Objetivos}

Os objetivos do trabalho serão detalhados em seguida, primeiro o objetivo geral em seguida os objetivos específicos.


\subsection{Objetivo Geral}

Criar uma ferramenta de apoio a criação de projetos de gamificação, onde a entrada de dados seja feita de maneira fácil, dinâmica e fluída e que dê suporte, inclusive, ao usuário que não têm conhecimento profundo sobre gamificação.

\subsection{Objetivos Específicos}

\begin{itemize}

\item  Selecionar um modelo de gamificação como base para ferramenta;
\item  Criar uma ferramenta de uso simples;
\item  Construir integração com o funifier;
\item  Realizar engenharia reversa de projetos criados no funifer.


\end{itemize}

\section{Questão de Pesquisa}

É possível desenvolver uma ferramenta de criação de projetos de gamificação capaz de produzir projetos de qualidade e acessível a todos?


\section{Motivação}

 Autores como \cite{chou2015actionable} e \cite{mcgonigal2011reality} acreditam que o uso da gamificação pode mudar vidas e até a sociedade em que estamos inseridos. Ela tem a capacidade de transformar empresas, escolas, hospitais e umas vasta gama de ambientes, basta ser aplicada corretamente em um contexto adequado. A motivação maior em desenvolver esse trabalho é poder oferecer as pessoas a oportunidade de criarem o seu próprio projeto e a partir desta iniciativa iniciarem uma mudança em suas vidas, ou na vida de alguém. Segundo McGonigal, os jogos são capazes de concentrar nossa energia em algo que somos bons e apreciamos fazer, com otimismo incansáveis, porque não fazer das atividades diárias um jogo?


\section{Metodologia}

\subsection{Classificação da Pesquisa}

A realização deste trabalho caracteriza uma pesquisa exploratória, que tem como caracteríticas buscar uma abordagem do fenômeno pelo levantamento de informações que poderão levar o pesquisador a conhecer mais a seu respeito \cite{}

\subsection{Referêncial Teórico}

O referencial teórico foi levantado utilizando livros físicos e digitais, artigos e alguns sites. O resultado das pesquisas efetuadas e que geramam o referencial teórico deste trabalho está disponível no capítulo 2.

\section{Estrutura da Monografia}

Este trabalho está fracionado em cinco capítulos, o capítulo 1 compreende a introdução, onde é feita uma contextualização sobre o tema e o trabalho, o capítulo 2, que corresponde ao referencial teórico e reflete as pesquisas realizadas para possibilitar a construção do trabalho, o capítulo 3, que contém a proposta do trabalho, capítulo 4, sobre engenharia de software, onde as lições aprendidas no curso são postas em práticas e capítulo 5, onde são feitas as considerações finais.
