
\chapter[Desenvolvimento da Proposta]{Desenvolvimento da Proposta}

\section{Descrição da ferramenta}

	A ferramenta será desenvolvida para plataforma web, onde o usuário irá construir o projeto de gamificação a partir das técnicas de gamificação ou partir das unidades principais, o usuário terá a liberdade de escolher como iniciar o projeto. 

\section{Proposta}

A ferramenta não se destinará apenas à estudiosos da área de gamificação, ela deverá também atender ao usuário comum, não especialista e guiá-lo de maneira simples para a construção do projeto que mais esteja alinhando com seus objetivos, para tanto será necessário durante o processo de construção informar ao usuário se o caminho que ele está seguindo é o correto e porque. Por isso a interação entre as técnicas de gamificação deverá ser mapeada, quando o usuário selecionar um conjunto de técnicas que não funcionam bem em conjunto ele será informado e pode realizar a troca. A ferramenta dará  liberdade ao usuário para criar o projeto a partir dos elementos contidos na ferramenta que mais mantenham motivado, mas não de maneira que o projeto fique fadado ao fracasso.

Para desenvolvimento da ferramenta prentende-se fazer uso do SCRUM com adaptações adequadas ao contexto do trabalho, será levantado o backlog de produto, backlog da sprint, escrita de histórias, detalhamento em tarefas e definição de critérios de aceitação, sprints com duração de duas semanas e restrospectiva ao fim das sprints.

\section{Sistema}

A ferramenta deve ser capaz de importar e exportar os modelos de projetos de gamificação. Apesar de ser uma ferramenta para construção de projetos de gamificações e não uma ferramenta gamificada a inserção de dados e o seu comportamento deve ser leve e fluído, de fácil uso  e interação por parte do usuário, para que não tenha os mesmos problemas encontrados na inserção de dados feitos através de planilhas excel e não desmotive o usuário logo no inicio do processo de construção do projeto de gamificação, seria contraditório construir uma ferramenta massante e arcaica dentro de um contexto de gamificação, seria iniciar todo o processo comentendo um erro. O usuário deve se sentir motivado desde o início do uso do sistema até o momento de conclusão do projeto e a usabilidade do sistema terá parte importanta nesse processo. A ferramenta deve ser capaz de permitir que o usuário vizualize o projeto como um todo.  A ferramenta irá dispor de modelos que servirão de base para a construção dos projetos, caso o usuário sinta necessidade, poderá usar um modelo de gamificação como exemplo. 



\subsection{Requisitos e funcionalidades}
Para desenvolvimendo do sistema foram levantadas histórias de usuário e critérios de aceitação. As histórias de usuários formarão o backlog do produto e os critérios de aceitação definem se o que será feito estará de acordo com os requisitos levantados, como usabilidade é uma parte muito importante do sistema a ser desenvolvido os critérios de aceitação servirão principalmente para validar se o sistema foi construído com a usabilidade esperada. As histórias levantadas e as suas respectivas pontuações se encontram na tabela (\ref{tab01}) . 



\begin{table}[!htpb]
\label{tab01}
\centering
\begin{tabular}{|p{1.5cm}|p{12cm}|} \hline

 Número & História de usuário \\ \hline

 1 & Eu, como usuário gostaria de realizar login no sistema, para que eu possa ter acesso aos meus projetos. \\ \hline

 2 & Eu, como usuário gostaria de mudar o status do projeto para público ou privado,  para ter privacidade quando necessário. \\ \hline

 3 & Eu, como usuário gostaria de importar e exportar os projetos de gamificação, para fazer uso em outras plataformas. \\ \hline

 4 & Eu, como usuário, gostaria de finalizar o projeto em um dia diferente do iniciado, para que eu possa modificar o projeto ao longo da construção. \\ \hline

 5 & Eu, como usuário, gostaria de receber um feedback quando o projeto de gamificação não estiver sendo construído de maneira consistente, para que eu possa realizar as modificações necessárias. \\ \hline

 6 & Eu, como usuário, gostaria de construir mais de um projeto, para montar uma galeria de projetos. \\ \hline

 7 & Eu, como usuário, gostaria de reutilizar um projeto finalizado, para aplicar a outro contexto. \\ \hline

 8 & Eu, como usuário, gostaria de escolher a partir de qual elemento inicio o projeto, para ter maior flexibilidade de idéias. \\ \hline

 9 & Eu, como usuário, gostaria de vizualiar uma descrição dos itens que compõem o projeto, para entender o que cada item significa. \\ \hline

 10 & Eu, como usuário, gostaria de vizualizar um tutorial interativo, para entender como funciona a ferramenta. \\ \hline

 11 & Eu, como usuário, gostaria de saber qual tipo de gamifição construi, para entender quais os aspectos serão motivados . \\ \hline

 12 & Eu, como usuário, gostaria de apagar um projeto, para não ter mais acesso ao mesmo. \\ \hline
\end{tabular}
\caption{Histórias de Usuário}
\end{table} 


\subsection{Lacunas da Proposta}

Quando algo novo é proposto ele está sujeito a falhas, as lacunas já identificadas na proposta estão listadas abaixo como perguntas.

\begin{itemize}
\item  A gamificação ultrapassa os limites dos aparelhos eletrônicos, não há necessidade real de nenhum aparato tecnologico para se criar uma gamificação, porque então a ferramenta será útil já que já existe material ensinando a criar gamificações?
	A ferramenta servirá para encurtar o processo de aprendizado sobre gamificação antes de criar o projeto e irá facilitar a criação do mesmo, visto que irá automatizar o processo.
\item  Como tornar a construção dos projetos divertidas?
	Uma interface intuitiva e centrada no usuário, com elementos simples de utilizar e com uma temática conhecida pelo usuário.
\item Como evitar ou reduzir o risco de fracasso ou ineficiência dos projetos de gamificação criados na ferramenta?
	O relacionamento estabelecido entre as técnicas de gamificação tem como objetivo reduzir a taxa de fracassos.
\item Como testar se o mapeamento e relacionamento das técnicas do não está sendo feito de maneira inadequada?
\item O principal foco da ferramenta é fazer uma inserção de dados menos desgastante ou criar gamificação de qualidade?
\item Até que ponto é possivel motivar os aspectos psicologicos intrisicamente ligados a gamificação através de uma ferramenta?

\end{itemize}

\subsection{Definição de tecnologias}

Para a construção da base de conhecimento será utilizado o Attribute-Logic Engine (ALE), no ALE cada estrutura representa um tipo e os tipos são organizados em hierarquia, onde as características de um tipo são herdadas pelos subtipos, o ALE é um sistema de representação e análise léxica, é um compilador de gramáticas que utiliza Prolog. O ALE será utilizado para montar a ontologia por trás da ferramenta, fará o raciocínio fazendo uso dos atributos.
A plataforma web será desenvolvida utilizando o paradigma orientado a objetos e arquitetura Model View Controller, em conjunto com o ALE, as linguagens cadidatas para o desenvolvimento da plataforma web são Ruby com o Framework Rails e Python, com o framework Django, juntamento com Angular JS, visto que a ferramenta precisa ser bastante dinâmica para o usuário. O versionamento do código será feita através do git, o banco de dados será o mysql e a ferramenta de análise de qualidade de código será definida juntamente com a definição da linguagem de desenvolvimento, para o gerenciamento das sprints será utilizada o waffle.io, onde ficará disposto o backlog do produto e o backlog da sprint, integrado ao repositório do git, onde as histórias serão divididas em tarefas e gerarão issues no github.


\section{Cronograma de desenvolvimento TCC1}


\begin{table}[!htpb]
\centering

% definindo o tamanho da fonte para small
% outros possíveis tamanhos: footnotesize, scriptsize
\begin{small} 
  
% redefinindo o espaçamento das colunas
\setlength{\tabcolsep}{3pt} 

% \cline é semelhante ao \hline, porém é possível indicar as colunas que terão essa a linha horizontal
% \multicolumn{10}{c|}{Meses} indica que dez colunas serão mescladas e a palavra Meses estará centralizada dentro delas.

\begin{tabular}{|c|c|c|c|c|}\hline
 & \multicolumn{4}{c|}{Período}\\ \cline{2-5}
\raisebox{1.5ex}{Etapa} & 15/02 à 22/02 & 01/03 a 31/03 & 04/04 à 22/05 & 16/05 à 15/06 \\ \hline

Apontar Tema & X & & & \\ \hline
Levantar Referencial Teórico & & X & & \\ \hline
Definir Metodologias & X & & & \\ \hline
Pré Projeto & & & X & \\ \hline
Mapeamento das Técnicas & & & X &  \\ \hline
Seleção dos atributos & & & X & \\ \hline
Escrita do TCC & & & & X \\ \hline

\end{tabular} 
\end{small}
\caption{Cronograma das atividades executadas}
\label{t_cronograma}
\end{table} 

\begin{itemize}
\item  \textbf {Apontar Tema:} Definir entre as propostas de temas qual seria escolhido,  definir qual a área e problema a ser atacado.
\item  \textbf {Levantar Referencial Teórico:} Após o apontamento do tema foram definidos os referênciais teóricos e iniciou-se a busca por material relevante sobre o problema a ser solucionado.
\item  \textbf {Definir Metodologias} Definir metodologia utilizada na construção do TCC.
\item  \textbf {Mapeamento das Técnicas de Gamificação:} Levantamento das técnicas de gamificação que compões o octalysis.
\item  \textbf {Seleção dos indicadores serão a composição das técnicas:} Definição de como seria a estrutura interna da técnica, seleção e valoração dos atributos.
\item  \textbf{SEscrita do TCC:} Escrita e formatação dos capítuslos do TCC.
\end{itemize}




