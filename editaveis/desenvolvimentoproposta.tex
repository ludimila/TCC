
\chapter[Desenvolvimento da Proposta]{Desenvolvimento da Proposta}

\section{Descrição da ferramenta}

	A ferramenta será desenvolvida para plataforma web, onde o usuário irá construir o projeto de gamificação a partir das técnicas de gamificação ou partir das unidades principais, o usuário terá a liberdade de escolher como iniciar o projeto. 

\section{Proposta}

A ferramenta não se destinará apenas à estudiosos da área de gamificação, ela deverá também atender ao usuário comum, não especialista e guiá-lo de maneira simples para a construção do projeto que mais esteja alinhando com seus objetivos, para tanto será necessário durante o processo de construção informar ao usuário se o caminho que ele está seguindo é o correto e porque. Por isso a interação entre as técnicas de gamificação deverá ser mapeada, quando o usuário selecionar um conjunto de técnicas que não funcionam bem em conjunto ele será informado e pode realizar a troca. A ferramenta dará  liberdade ao usuário para criar o projeto a partir dos elementos contidos na ferramenta que mais mantenham motivado, mas não de maneira que o projeto fique fadado ao fracasso.

Para desenvolvimento da ferramenta prentende-se fazer uso do SCRUM com adaptações adequadas ao contexto do trabalho, será levantado o backlog de produto, backlog da sprint, escrita de histórias, detalhamento em tarefas e definição de critérios de aceitação, sprints com duração de duas semanas e restrospectiva ao fim das sprints.

\section{Sistema}

Para desenvolvimendo do sistema foram levantadas histórias de usuário e critérios de aceitação. As histórias de usuários formarão o backlog do produto e os critérios de aceitação definem se o que será feito estará de acordo com os requisitos levantados, como usabilidade é uma parte muito importante do sistema a ser desenvolvido os critérios de aceitação servirão principalmente para validar se o sistema foi construído com a usabilidade esperada. As histórias foram pontuadas de acordo com a sequência de Fibonacci e se a história ficasse com mais de 13 pontos era refatorada e divida em duas histórias menores. As histórias levantadas e as suas respectivas pontuações se encontram na tabela X. A história será considerada pronta 

\begin{tabular}{|l||l|l|l|l|}

\hline
\multicolumn{2}{|c}{\textbf{Histórias de Usuário}}\\ \hline
Pontuação & Histórias \\
\hline
1 &Eu, como usuário gostaria de realizar login no sistema, para que eu possa ter acesso aos meus projetos.\\
\hline
2 &Eu, como usuário gostaria de mudar o status do projeto para público ou privado,  para ter privacidade quando necessário.\\
\hline
3 & Eu, como usuário, gostaria de compartilhar o projeto, para que outras pessoas possam ter acesso ao mesmo.\\
\hline
4 &  Eu, como usuário, gostaria de finalizar o projeto em um dia diferente do iniciado, para que eu possa modificar o projeto ao longo da construção. \\
\hline
5 & Eu, como usuário, gostaria de receber um feedback quando o projeto de gamificação não estiver sendo construído de maneira consistente, para que eu possa realizar as modificações necessárias.\\
\hline
6 &Eu, como usuário, gostaria de construir mais de um projeto, para montar uma galeria de projetos.\\
\hline
7 & Eu, como usuário, gostaria de reutilizar um projeto finalizado, para aplicar a outro contexto.\\
\hline
8 & Eu, como usuário, gostaria de escolher a partir de qual elemento inicio o projeto, para ter maior flexibilidade de idéias.\\
\hline
9 & Eu, como usuário, gostaria de vizualiar uma descrição dos itens que compõem o projeto, para entender o que cada item significa.\\
\hline
10 & Eu, como usuário, gostaria de vizualizar um tutorial interativo, para entender como funciona a ferramenta.\\
\hline
11 & Eu, como usuário, gostaria de saber qual tipo de gamifição construi, para entender quais os aspectos irei motivar .\\
\hline
12 & Eu, como usuário, gostaria de apagar um projeto, para ... .\\
\hline
\end{tabular}

\subsection{Critérios de Aceitação}


\subsection{Protótipos}

	
\subsection{Definição de tecnologias}

\section{Cronograma de desenvolvimento}