\chapter[Resultados]{RESULTADOS}

Este capítulo trata das conclusões e trabalhos futuros propostos. Ele corresponde a última das práticas de engenharia, citadas no capítulo anterior, examinar resultados.

As histórias executadas podem ser vistas na tabela abaixo.

\begin{table}[!htpb]
\centering
\begin{tabular}{|p{1.5cm}|p{12cm}|} \hline

 Número & História de usuário \\ \hline

 1 & Eu, como usuário gostaria de realizar login no sistema, para que eu possa ter acesso aos meus projetos. \\ \hline

 3 & Eu, como usuário gostaria  de exportar os projetos de gamificação, para fazer uso em outras plataformas. \\ \hline

 4 & Eu, como usuário, gostaria de finalizar o projeto em um dia diferente do iniciado, para que eu possa modificar o projeto ao longo da construção. \\ \hline

 6 & Eu, como usuário, gostaria de construir mais de um projeto, para montar uma galeria de projetos. \\ \hline

 7 & Eu, como usuário, gostaria de reutilizar um projeto finalizado, para aplicar a outro contexto. \\ \hline

 9 & Eu, como usuário, gostaria de visualizar uma descrição dos itens que compõem o projeto, para entender o que cada item significa. \\ \hline

 11 & Eu, como usuário, gostaria de saber qual tipo de gamificação construí, para entender quais os aspectos serão motivados . \\ \hline

 12 & Eu, como usuário, gostaria de apagar um projeto, para não ter mais acesso ao mesmo. \\ \hline

 13 & Eu, como usuário, gostaria de visualizar um relatorio do projeto, para ter acesso depois de finalizar um projeto. \\ \hline


\end{tabular}
\caption{Histórias Executadas\label{feitas}
}
\end{table} 

\section {Examinar resultados}


Com o foco de entregar uma ferramenta capaz de apoiar a criação de projetos de gamificação o escopo foi adaptado. Ao finalizar o desenvolvimento é possível responder a questão de pesquisa feita antes de iniciar o desenvolvimento. A pergunta era \textit{É possível desenvolver uma ferramenta de apoio a criação de projetos de gamificação capaz de produzir projetos de gamificação?"}. Sim é possível, a ferramenta criada auxilia no processo de construção de um projeto de gamificação.


A plataforma desenvolvida utiliza a extensão proposta ao \textit{framework} Octalysis. Fornece \textit{feedback} ao usuário através dos relatórios as informações quanto o projeto de gamificação construído.

O arquivo .CSV gerado facilita a entrada de dados para o Funifier, visto que foi criado espelhando as planilhas já utilizadas. O relatório .PDF também fornece imformações pertinentes. 

A extensão proposta ao Octalysis faz parte da proposta de metodologia para avaliação da gamificação em jogos, que rendeu uma apresentação e publicação de um artigo B2 no SBIE 2016.

Para trabalhos futuros pretende-se realizar as quatro histórias que não foram colocadas em prática. As histórias não executadas estão expostas abaixo.

\begin{table}[!htpb]
\centering
\begin{tabular}{|p{1.5cm}|p{12cm}|} \hline

 2 & Eu, como usuário gostaria de mudar o status do projeto para público ou privado,  para ter privacidade quando necessário. \\ \hline

 5 & Eu, como usuário, gostaria de receber um \textit{feedback} quando o projeto de gamificação não estiver sendo construído de maneira consistente, para que eu possa realizar as modificações necessárias. \\ \hline

 8 & Eu, como usuário, gostaria de escolher a partir de qual elemento inicio o projeto, para ter maior flexibilidade de idéias. \\ \hline

 10 & Eu, como usuário, gostaria de visualizar um tutorial interativo, para entender como funciona a ferramenta. \\ \hline

 
\end{tabular}
\caption{Histórias não executadas\label{nfeitas}
}
\end{table} 





